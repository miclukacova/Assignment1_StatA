% Options for packages loaded elsewhere
\PassOptionsToPackage{unicode}{hyperref}
\PassOptionsToPackage{hyphens}{url}
%
\documentclass[
]{article}
\usepackage{amsmath,amssymb}
\usepackage{iftex}
\ifPDFTeX
  \usepackage[T1]{fontenc}
  \usepackage[utf8]{inputenc}
  \usepackage{textcomp} % provide euro and other symbols
\else % if luatex or xetex
  \usepackage{unicode-math} % this also loads fontspec
  \defaultfontfeatures{Scale=MatchLowercase}
  \defaultfontfeatures[\rmfamily]{Ligatures=TeX,Scale=1}
\fi
\usepackage{lmodern}
\ifPDFTeX\else
  % xetex/luatex font selection
\fi
% Use upquote if available, for straight quotes in verbatim environments
\IfFileExists{upquote.sty}{\usepackage{upquote}}{}
\IfFileExists{microtype.sty}{% use microtype if available
  \usepackage[]{microtype}
  \UseMicrotypeSet[protrusion]{basicmath} % disable protrusion for tt fonts
}{}
\makeatletter
\@ifundefined{KOMAClassName}{% if non-KOMA class
  \IfFileExists{parskip.sty}{%
    \usepackage{parskip}
  }{% else
    \setlength{\parindent}{0pt}
    \setlength{\parskip}{6pt plus 2pt minus 1pt}}
}{% if KOMA class
  \KOMAoptions{parskip=half}}
\makeatother
\usepackage{xcolor}
\usepackage[margin=1in]{geometry}
\usepackage{color}
\usepackage{fancyvrb}
\newcommand{\VerbBar}{|}
\newcommand{\VERB}{\Verb[commandchars=\\\{\}]}
\DefineVerbatimEnvironment{Highlighting}{Verbatim}{commandchars=\\\{\}}
% Add ',fontsize=\small' for more characters per line
\usepackage{framed}
\definecolor{shadecolor}{RGB}{248,248,248}
\newenvironment{Shaded}{\begin{snugshade}}{\end{snugshade}}
\newcommand{\AlertTok}[1]{\textcolor[rgb]{0.94,0.16,0.16}{#1}}
\newcommand{\AnnotationTok}[1]{\textcolor[rgb]{0.56,0.35,0.01}{\textbf{\textit{#1}}}}
\newcommand{\AttributeTok}[1]{\textcolor[rgb]{0.13,0.29,0.53}{#1}}
\newcommand{\BaseNTok}[1]{\textcolor[rgb]{0.00,0.00,0.81}{#1}}
\newcommand{\BuiltInTok}[1]{#1}
\newcommand{\CharTok}[1]{\textcolor[rgb]{0.31,0.60,0.02}{#1}}
\newcommand{\CommentTok}[1]{\textcolor[rgb]{0.56,0.35,0.01}{\textit{#1}}}
\newcommand{\CommentVarTok}[1]{\textcolor[rgb]{0.56,0.35,0.01}{\textbf{\textit{#1}}}}
\newcommand{\ConstantTok}[1]{\textcolor[rgb]{0.56,0.35,0.01}{#1}}
\newcommand{\ControlFlowTok}[1]{\textcolor[rgb]{0.13,0.29,0.53}{\textbf{#1}}}
\newcommand{\DataTypeTok}[1]{\textcolor[rgb]{0.13,0.29,0.53}{#1}}
\newcommand{\DecValTok}[1]{\textcolor[rgb]{0.00,0.00,0.81}{#1}}
\newcommand{\DocumentationTok}[1]{\textcolor[rgb]{0.56,0.35,0.01}{\textbf{\textit{#1}}}}
\newcommand{\ErrorTok}[1]{\textcolor[rgb]{0.64,0.00,0.00}{\textbf{#1}}}
\newcommand{\ExtensionTok}[1]{#1}
\newcommand{\FloatTok}[1]{\textcolor[rgb]{0.00,0.00,0.81}{#1}}
\newcommand{\FunctionTok}[1]{\textcolor[rgb]{0.13,0.29,0.53}{\textbf{#1}}}
\newcommand{\ImportTok}[1]{#1}
\newcommand{\InformationTok}[1]{\textcolor[rgb]{0.56,0.35,0.01}{\textbf{\textit{#1}}}}
\newcommand{\KeywordTok}[1]{\textcolor[rgb]{0.13,0.29,0.53}{\textbf{#1}}}
\newcommand{\NormalTok}[1]{#1}
\newcommand{\OperatorTok}[1]{\textcolor[rgb]{0.81,0.36,0.00}{\textbf{#1}}}
\newcommand{\OtherTok}[1]{\textcolor[rgb]{0.56,0.35,0.01}{#1}}
\newcommand{\PreprocessorTok}[1]{\textcolor[rgb]{0.56,0.35,0.01}{\textit{#1}}}
\newcommand{\RegionMarkerTok}[1]{#1}
\newcommand{\SpecialCharTok}[1]{\textcolor[rgb]{0.81,0.36,0.00}{\textbf{#1}}}
\newcommand{\SpecialStringTok}[1]{\textcolor[rgb]{0.31,0.60,0.02}{#1}}
\newcommand{\StringTok}[1]{\textcolor[rgb]{0.31,0.60,0.02}{#1}}
\newcommand{\VariableTok}[1]{\textcolor[rgb]{0.00,0.00,0.00}{#1}}
\newcommand{\VerbatimStringTok}[1]{\textcolor[rgb]{0.31,0.60,0.02}{#1}}
\newcommand{\WarningTok}[1]{\textcolor[rgb]{0.56,0.35,0.01}{\textbf{\textit{#1}}}}
\usepackage{longtable,booktabs,array}
\usepackage{calc} % for calculating minipage widths
% Correct order of tables after \paragraph or \subparagraph
\usepackage{etoolbox}
\makeatletter
\patchcmd\longtable{\par}{\if@noskipsec\mbox{}\fi\par}{}{}
\makeatother
% Allow footnotes in longtable head/foot
\IfFileExists{footnotehyper.sty}{\usepackage{footnotehyper}}{\usepackage{footnote}}
\makesavenoteenv{longtable}
\usepackage{graphicx}
\makeatletter
\def\maxwidth{\ifdim\Gin@nat@width>\linewidth\linewidth\else\Gin@nat@width\fi}
\def\maxheight{\ifdim\Gin@nat@height>\textheight\textheight\else\Gin@nat@height\fi}
\makeatother
% Scale images if necessary, so that they will not overflow the page
% margins by default, and it is still possible to overwrite the defaults
% using explicit options in \includegraphics[width, height, ...]{}
\setkeys{Gin}{width=\maxwidth,height=\maxheight,keepaspectratio}
% Set default figure placement to htbp
\makeatletter
\def\fps@figure{htbp}
\makeatother
\setlength{\emergencystretch}{3em} % prevent overfull lines
\providecommand{\tightlist}{%
  \setlength{\itemsep}{0pt}\setlength{\parskip}{0pt}}
\setcounter{secnumdepth}{-\maxdimen} % remove section numbering
\ifLuaTeX
  \usepackage{selnolig}  % disable illegal ligatures
\fi
\usepackage{bookmark}
\IfFileExists{xurl.sty}{\usepackage{xurl}}{} % add URL line breaks if available
\urlstyle{same}
\hypersetup{
  pdftitle={Assignment1},
  hidelinks,
  pdfcreator={LaTeX via pandoc}}

\title{Assignment1}
\author{}
\date{\vspace{-2.5em}2024-12-07}

\begin{document}
\maketitle

\section{Part 1}\label{part-1}

\subsection{5.}\label{section}

We simulate the data. We simulate \(10^5\) data points form the
multivariate normal distribution with mean
\(\begin{pmatrix} 0 \\ 0 \\ 0 \end{pmatrix}\) and variance
\(\begin{pmatrix} 1 & 0.25 & 0.5 \\ 0.25 & 1 & 0.5 \\ 0.5 & 0.5 & 1 \end{pmatrix}\).

\begin{Shaded}
\begin{Highlighting}[]
\FunctionTok{set.seed}\NormalTok{(}\DecValTok{7878}\NormalTok{)}
\NormalTok{n }\OtherTok{\textless{}{-}} \DecValTok{10}\SpecialCharTok{\^{}}\DecValTok{5}
\NormalTok{sigma }\OtherTok{\textless{}{-}} \FunctionTok{matrix}\NormalTok{(}\FunctionTok{c}\NormalTok{(}\DecValTok{1}\NormalTok{, }\FloatTok{0.25}\NormalTok{, }\FloatTok{0.5}\NormalTok{, }\FloatTok{0.25}\NormalTok{, }\DecValTok{1}\NormalTok{, }\FloatTok{0.5}\NormalTok{, }\FloatTok{0.5}\NormalTok{, }\FloatTok{0.5}\NormalTok{, }\DecValTok{1}\NormalTok{), }\AttributeTok{nrow =} \DecValTok{3}\NormalTok{)}
\NormalTok{X }\OtherTok{\textless{}{-}} \FunctionTok{mvrnorm}\NormalTok{(n, }\FunctionTok{c}\NormalTok{(}\DecValTok{0}\NormalTok{,}\DecValTok{0}\NormalTok{, }\DecValTok{0}\NormalTok{), sigma)}
\end{Highlighting}
\end{Shaded}

We will illustrate that the conditional distribution of
\(\begin{pmatrix} X_1 \\ X_2 \end{pmatrix} \vert X_3 = x_3\) is as
expected by conditioning on \(|X_3| = 0\). To approximate this (\(X_3\)
has a continuous distribution), we find the rows of \(X\) for which
\(|X_3| \leq 0.05\):

\begin{Shaded}
\begin{Highlighting}[]
\NormalTok{cond }\OtherTok{\textless{}{-}} \FunctionTok{which}\NormalTok{(}\FunctionTok{abs}\NormalTok{(X[,}\DecValTok{3}\NormalTok{]) }\SpecialCharTok{\textless{}} \FloatTok{0.05}\NormalTok{)}
\end{Highlighting}
\end{Shaded}

We calculate the correlation of \(X_1\) and \(X_2\) for those \(X_1\)
and \(X_2\) for which \(|X_3| \leq 0.05\).

\begin{Shaded}
\begin{Highlighting}[]
\FunctionTok{cor}\NormalTok{(X[cond,}\DecValTok{1}\NormalTok{], X[cond,}\DecValTok{2}\NormalTok{])}
\end{Highlighting}
\end{Shaded}

\begin{verbatim}
## [1] 0.001381612
\end{verbatim}

Which is as expected close to \(0\). This is different from the
correlation of \(X_1\) and \(X_2\) in general, which is by construction
\(\approx 0.25\):

\begin{Shaded}
\begin{Highlighting}[]
\FunctionTok{cor}\NormalTok{(X[,}\DecValTok{1}\NormalTok{], X[,}\DecValTok{2}\NormalTok{])}
\end{Highlighting}
\end{Shaded}

\begin{verbatim}
## [1] 0.2495407
\end{verbatim}

To visualize the conditional distribution of \(X_1\) and \(X_2\) we
create the two scatterplot below. In the unconditional distribution the
two variables are positively correlated, while they in the conditional
distribution are uncorrelated, and thus (since they are normal)
independent.

\begin{center}\includegraphics{Aflevering_1_pdf_files/figure-latex/unnamed-chunk-5-1} \end{center}

As a last illustration, we have made histograms of the marginal
distributions of the two random variables, conditional on \(X_3 = 0\)
and unconditionally. The marginal distribution of both variables
unconditionally is \(\mathcal{N}(0,1)\) and conditionally on
\(X_3 = 0\), it is, with use of our calculations from the previous
exercises, \(\mathcal{N}(0, 0.75)\).

\begin{verbatim}
## Warning: The dot-dot notation (`..density..`) was deprecated in ggplot2 3.4.0.
## i Please use `after_stat(density)` instead.
## This warning is displayed once every 8 hours.
## Call `lifecycle::last_lifecycle_warnings()` to see where this warning was
## generated.
\end{verbatim}

\begin{verbatim}
## `stat_bin()` using `bins = 30`. Pick better value with `binwidth`.
## `stat_bin()` using `bins = 30`. Pick better value with `binwidth`.
## `stat_bin()` using `bins = 30`. Pick better value with `binwidth`.
## `stat_bin()` using `bins = 30`. Pick better value with `binwidth`.
\end{verbatim}

\begin{center}\includegraphics{Aflevering_1_pdf_files/figure-latex/unnamed-chunk-6-1} \end{center}

Notice also how the variance reduces as we condition, this is also to be
expected as we gain further information.

In example 2.5 we showed that the variance matrix of the conditional
distribution does not depend on the value of the conditioning variable.
Having thus showed that the \(X_1| X_3 = 0\) and \$ X\_2 \textbar{} X\_3
= 0\$ are independent and with the expected variance, this will also be
true for all other values of \(X_3\). (\emph{er det rigtigt???})

All in all the conditional distribution of
\(\begin{pmatrix} X_1 \\ X_2 \end{pmatrix} \vert X_3\) behaves very much
as expected.

\section{Part 2}\label{part-2}

\begin{Shaded}
\begin{Highlighting}[]
\FunctionTok{load}\NormalTok{(}\StringTok{"assignment2024{-}1.Rdata"}\NormalTok{)}
\end{Highlighting}
\end{Shaded}

Data plots:

\begin{center}\includegraphics{Aflevering_1_pdf_files/figure-latex/unnamed-chunk-8-1} \end{center}

Model fits:

\begin{Shaded}
\begin{Highlighting}[]
\NormalTok{fit1 }\OtherTok{\textless{}{-}} \FunctionTok{lmer}\NormalTok{(Liking }\SpecialCharTok{\textasciitilde{}}\NormalTok{ Product }\SpecialCharTok{+}\NormalTok{ (}\DecValTok{1}\SpecialCharTok{|}\NormalTok{Participant) }\SpecialCharTok{+}\NormalTok{ (}\DecValTok{1}\SpecialCharTok{|}\NormalTok{Class), }\AttributeTok{data=}\NormalTok{likingdata)}
\NormalTok{fit2 }\OtherTok{\textless{}{-}} \FunctionTok{lmer}\NormalTok{(Liking }\SpecialCharTok{\textasciitilde{}}\NormalTok{ ProdVersion }\SpecialCharTok{+}\NormalTok{ ProdType }\SpecialCharTok{+}\NormalTok{ (}\DecValTok{1}\SpecialCharTok{|}\NormalTok{Participant) }\SpecialCharTok{+}\NormalTok{ (}\DecValTok{1}\SpecialCharTok{|}\NormalTok{Class), }\AttributeTok{data=}\NormalTok{likingdata)}
\end{Highlighting}
\end{Shaded}

\subsection{1.}\label{section-1}

The statistical model from fit1, can be described as:

\[Y_{ijk} = \beta_{i} + B_{j}^{\text{par}} + B_{k}^{\text{class}} + \epsilon_{ijk}\]

Where we use triple indexing. \(i=1,\ldots, 6\) corresponds to the \(6\)
different products, \(j=1,\ldots,75\) corresponds to the \(75\)
different participants, \(k=1,\ldots,5\) corresponds to the \(5\)
different classes.

Where
\(B^{\text{par}} \sim \mathcal{N}_{75}(0, \tau_{\text{par}}^2 I_{75})\),
\(B^{\text{class}} \sim \mathcal{N}_{5}(0, \tau_{\text{class}}^2 I_{5})\)
and \(\epsilon \sim \mathcal{N}_{450}(0, \sigma^2 I_{450})\) are all
independent.

Participant and class are considered random effects, since we are not
interested in the specific participants or classes as such, but rather
as representatives of the population of children and classes.

Product is included as a fixed effect since we are interested in
investigating the child friendliness of the different products.

\subsection{2.}\label{section-2}

The formula for the correlation for two random variables \(X\) and \(Y\)
is \(\frac{Cov(X,Y)}{\sqrt{(VY\cdot VY)}}\). By the independence of the
random variables in the statistical model for fit1 we get that
\(VY_{ijk}=\tau_{par}^2+\tau_{class}^2+ \sigma^2\). for all \(i,j\) and
\(k\). We now consider the two different cases:

If we look at the correlation between two observations for the same
participants \(Y_{ijk}\) and \(Y_{ljk}\), we have independence between
all random variable the bilinear properties of the covariance that: \[
Cov(Y_{ijk},Y_{ljk}) = Cov
\]

\subsection{3.}\label{section-3}

The factor \texttt{Product} is the interaction of the two factors
\texttt{ProdVersion} and \texttt{ProdType}, therefore the subspace
spanned by \texttt{ProdVersion} and \texttt{ProdType} is included in the
subspace spanned by \texttt{Product}, and fit2 is thus a submodel of
fit1. In other words if we know the \texttt{Product} we also know the
\texttt{ProdVersion} and \texttt{ProdType}.

In fit1 we estimate \(6\) different fixed effects parameters, 1
intercept parameter, and then \(5\) additional parameters for each
additional interaction level between \texttt{ProdVersion} and
\texttt{ProdType}. In fit2 we estimate \(4\) fixed effect parameters,
\(1\) intercept parameter, \(2\) for each additional
\texttt{ProdVersion} level and \(1\) for the last level of
\texttt{ProdType}. We assume no interaction effects between the two
factors \texttt{ProdVersion} and \texttt{ProdType} in fit2.

Letting \(L_0\) denote the subspace of \(\mathbb{R}^n\) spanned by the
model matrix from fit2 and \(L_X\) denote the subspace of
\(\\mathbb{R}^n\) spanned by the model matrix from fit1, we can test the
hypothesis of \(EY \in L_0 \subseteq L_X\) with the likelihood ratio
statistic. The test relies on asymptotic results which we use without
further arguments. We perform the test by use of the anova command:

\begin{Shaded}
\begin{Highlighting}[]
\FunctionTok{anova}\NormalTok{(fit1,fit2)}
\end{Highlighting}
\end{Shaded}

\begin{verbatim}
## refitting model(s) with ML (instead of REML)
\end{verbatim}

\begin{verbatim}
## Data: likingdata
## Models:
## fit2: Liking ~ ProdVersion + ProdType + (1 | Participant) + (1 | Class)
## fit1: Liking ~ Product + (1 | Participant) + (1 | Class)
##      npar    AIC    BIC  logLik deviance  Chisq Df Pr(>Chisq)
## fit2    7 1777.8 1806.6 -881.91   1763.8                     
## fit1    9 1781.1 1818.1 -881.56   1763.1 0.7031  2     0.7036
\end{verbatim}

With a significance level of 0.05 we can most certainly not reject the
null hypothesis, and we can conclude that the interaction factor
\texttt{product} does not improve the model fit significantly.

\emph{Jeg tænker at siden at p-værdien er så høj at vi ikke gider at
simulere?}

\begin{Shaded}
\begin{Highlighting}[]
\DocumentationTok{\#\# Simulated p{-}value in test for TVset}
\CommentTok{\#sim12 \textless{}{-} pbkrtest::PBmodcomp(fit1, fit2, nsim=2000, seed=967)}
\CommentTok{\#}
\DocumentationTok{\#\# Extract simulated LRTs}
\CommentTok{\#LRT\_12 \textless{}{-} as.numeric(sim12$ref)}
\CommentTok{\#}
\DocumentationTok{\#\# Density for chi{-}square with df=1}
\CommentTok{\#dchisq2 \textless{}{-} function(x) dchisq(x,df=2)}
\CommentTok{\#}
\DocumentationTok{\#\# Histogram with overlaid density}
\CommentTok{\#data.frame(LRT\_12 = LRT\_12) |\textgreater{} }
\CommentTok{\#  ggplot(aes(x = LRT\_12)) + }
\CommentTok{\#  geom\_histogram(aes(y = ..density..), breaks=seq(0,18,0.5), color="black", fill="white") +}
\CommentTok{\#  geom\_function(fun = dchisq2, colour = "red", xlim=c(0.12,15), linewidth=1) +}
\CommentTok{\#  xlab("LRT") + ylab("Density") + ggtitle("Test for ProdVersion effect") +}
\CommentTok{\#  geom\_vline(xintercept=0.7031, color="blue",linewidth=1, linetype="dashed")}
\end{Highlighting}
\end{Shaded}

\begin{Shaded}
\begin{Highlighting}[]
\CommentTok{\#ggplot()+}
\CommentTok{\#  geom\_histogram(aes(x = (1 {-} pchisq(LRT\_12, df = 2)), y = ..density..))+}
\CommentTok{\#  geom\_hline(yintercept = 1)}
\end{Highlighting}
\end{Shaded}

\subsection{4.}\label{section-4}

\subsection{5.}\label{section-5}

\subsection{6.}\label{section-6}

\subsubsection{Simulating from the
t-distribution}\label{simulating-from-the-t-distribution}

The mean of the \(t\)-distribution is already \(0\). The variance of the
\(t\)-distribution with \(\nu\) degrees of freedom is
\(\frac{\nu}{\nu-2} = \frac{3}{3-2} = 3\). In order to achieve a
variance of \(\sigma^2\) we would therefore need to scale
\(X \sim t(3)\) with
\[\sigma^2= V(c \cdot X) = c^2 3 \Leftrightarrow c = {\frac{\sigma}{\sqrt3}}\]
We define the scaling factors

\begin{Shaded}
\begin{Highlighting}[]
\NormalTok{tauP }\OtherTok{\textless{}{-}} \DecValTok{1}\NormalTok{; tauC }\OtherTok{\textless{}{-}} \DecValTok{1}\NormalTok{; sigma }\OtherTok{\textless{}{-}} \DecValTok{1}
\NormalTok{c\_par }\OtherTok{\textless{}{-}}\NormalTok{ tauP }\SpecialCharTok{/} \FunctionTok{sqrt}\NormalTok{(}\DecValTok{3}\NormalTok{)}
\NormalTok{c\_class }\OtherTok{\textless{}{-}}\NormalTok{ tauC }\SpecialCharTok{/} \FunctionTok{sqrt}\NormalTok{(}\DecValTok{3}\NormalTok{)}
\NormalTok{c\_eps }\OtherTok{\textless{}{-}}\NormalTok{ sigma }\SpecialCharTok{/} \FunctionTok{sqrt}\NormalTok{(}\DecValTok{3}\NormalTok{)}
\end{Highlighting}
\end{Shaded}

We modify the simulation from question 5 to draw from the
t-distribution. The variables drawn are scaled by the scaling factors
defined above.

\begin{Shaded}
\begin{Highlighting}[]
\NormalTok{M }\OtherTok{\textless{}{-}} \DecValTok{2000}
\NormalTok{n\_eps }\OtherTok{\textless{}{-}} \FunctionTok{nrow}\NormalTok{(likingdata)}
\NormalTok{n\_par }\OtherTok{\textless{}{-}} \FunctionTok{unique}\NormalTok{(likingdata}\SpecialCharTok{$}\NormalTok{Participant) }\SpecialCharTok{|\textgreater{}} \FunctionTok{length}\NormalTok{()}
\NormalTok{n\_class }\OtherTok{\textless{}{-}} \FunctionTok{unique}\NormalTok{(likingdata}\SpecialCharTok{$}\NormalTok{Class) }\SpecialCharTok{|\textgreater{}} \FunctionTok{length}\NormalTok{()}

\NormalTok{X }\OtherTok{\textless{}{-}}\NormalTok{ fit2 }\SpecialCharTok{|\textgreater{}} \FunctionTok{model.matrix}\NormalTok{()}
\NormalTok{Z }\OtherTok{\textless{}{-}} \FunctionTok{getME}\NormalTok{(fit2, }\StringTok{"Z"}\NormalTok{)}
\NormalTok{beta }\OtherTok{\textless{}{-}}\NormalTok{ (}\FunctionTok{summary}\NormalTok{(fit2) }\SpecialCharTok{|\textgreater{}} \FunctionTok{coef}\NormalTok{())[,}\DecValTok{1}\NormalTok{]}
\NormalTok{tauP }\OtherTok{\textless{}{-}} \FunctionTok{data.frame}\NormalTok{(}\FunctionTok{VarCorr}\NormalTok{(fit2))[}\DecValTok{1}\NormalTok{,}\DecValTok{5}\NormalTok{]}
\NormalTok{tauC }\OtherTok{\textless{}{-}} \FunctionTok{data.frame}\NormalTok{(}\FunctionTok{VarCorr}\NormalTok{(fit2))[}\DecValTok{2}\NormalTok{,}\DecValTok{5}\NormalTok{]}
\NormalTok{sigma }\OtherTok{\textless{}{-}} \FunctionTok{data.frame}\NormalTok{(}\FunctionTok{VarCorr}\NormalTok{(fit2))[}\DecValTok{3}\NormalTok{,}\DecValTok{5}\NormalTok{]}
\NormalTok{deltasim2 }\OtherTok{\textless{}{-}} \FunctionTok{matrix}\NormalTok{(}\ConstantTok{NA}\NormalTok{,M,}\DecValTok{3}\NormalTok{)}

\ControlFlowTok{for}\NormalTok{ (i }\ControlFlowTok{in} \DecValTok{1}\SpecialCharTok{:}\NormalTok{M)\{}
\NormalTok{  B1 }\OtherTok{\textless{}{-}} \FunctionTok{rt}\NormalTok{(}\AttributeTok{n =}\NormalTok{ n\_par, }\AttributeTok{df =} \DecValTok{3}\NormalTok{) }\SpecialCharTok{*}\NormalTok{ c\_par}
\NormalTok{  B2 }\OtherTok{\textless{}{-}} \FunctionTok{rt}\NormalTok{(}\AttributeTok{n =}\NormalTok{ n\_class, }\AttributeTok{df =} \DecValTok{3}\NormalTok{) }\SpecialCharTok{*}\NormalTok{ c\_class}
\NormalTok{  eps }\OtherTok{\textless{}{-}} \FunctionTok{rt}\NormalTok{(}\AttributeTok{n =}\NormalTok{ n\_eps, }\AttributeTok{df =} \DecValTok{3}\NormalTok{) }\SpecialCharTok{*}\NormalTok{ c\_eps}
\NormalTok{  B }\OtherTok{\textless{}{-}} \FunctionTok{c}\NormalTok{(B1,B2)}
\NormalTok{  y }\OtherTok{\textless{}{-}}\NormalTok{ X }\SpecialCharTok{\%*\%}\NormalTok{ beta }\SpecialCharTok{+}\NormalTok{ Z }\SpecialCharTok{\%*\%}\NormalTok{ B }\SpecialCharTok{+}\NormalTok{ eps}
\NormalTok{  y }\OtherTok{\textless{}{-}}\NormalTok{ y }\SpecialCharTok{|\textgreater{}} \FunctionTok{as.numeric}\NormalTok{() }\CommentTok{\# NB. This seems to be necessary}
\NormalTok{  lmm2 }\OtherTok{\textless{}{-}} \FunctionTok{lmer}\NormalTok{(y }\SpecialCharTok{\textasciitilde{}}\NormalTok{ ProdVersion }\SpecialCharTok{+}\NormalTok{ ProdType }\SpecialCharTok{+}\NormalTok{ (}\DecValTok{1}\SpecialCharTok{|}\NormalTok{Participant) }\SpecialCharTok{+}\NormalTok{ (}\DecValTok{1}\SpecialCharTok{|}\NormalTok{Class), }\AttributeTok{data=}\NormalTok{likingdata)}
\NormalTok{  deltasim2[i,}\DecValTok{1}\NormalTok{] }\OtherTok{\textless{}{-}} \FunctionTok{fixef}\NormalTok{(lmm2)[}\DecValTok{4}\NormalTok{]}
\NormalTok{  deltasim2[i,}\DecValTok{2}\SpecialCharTok{:}\DecValTok{3}\NormalTok{] }\OtherTok{\textless{}{-}}\NormalTok{ (lmm2 }\SpecialCharTok{|\textgreater{}} \FunctionTok{confint}\NormalTok{(}\AttributeTok{method=}\StringTok{"Wald"}\NormalTok{))[}\DecValTok{7}\NormalTok{,]}
\NormalTok{\}}

\NormalTok{deltasim2 }\OtherTok{\textless{}{-}}\NormalTok{ deltasim2 }\SpecialCharTok{|\textgreater{}} \FunctionTok{data.frame}\NormalTok{()}
\FunctionTok{names}\NormalTok{(deltasim2) }\OtherTok{\textless{}{-}} \FunctionTok{c}\NormalTok{(}\StringTok{"est"}\NormalTok{,}\StringTok{"lower"}\NormalTok{,}\StringTok{"upper"}\NormalTok{)}
\end{Highlighting}
\end{Shaded}

We calculate the bias:

\begin{Shaded}
\begin{Highlighting}[]
\FunctionTok{mean}\NormalTok{(deltasim2}\SpecialCharTok{$}\NormalTok{est }\SpecialCharTok{{-}} \FunctionTok{fixef}\NormalTok{(fit2)[}\DecValTok{4}\NormalTok{]) }\SpecialCharTok{|\textgreater{}}\NormalTok{ knitr}\SpecialCharTok{::}\FunctionTok{kable}\NormalTok{(}\AttributeTok{col.names =} \StringTok{" "}\NormalTok{)}
\end{Highlighting}
\end{Shaded}

\begin{longtable}[]{@{}r@{}}
\toprule\noalign{}
\endhead
\bottomrule\noalign{}
\endlastfoot
-0.000313 \\
\end{longtable}

And the coverage:

\begin{Shaded}
\begin{Highlighting}[]
\FunctionTok{mean}\NormalTok{(deltasim2}\SpecialCharTok{$}\NormalTok{lower }\SpecialCharTok{\textless{}=} \FunctionTok{fixef}\NormalTok{(fit2)[}\DecValTok{4}\NormalTok{] }\SpecialCharTok{\&} \FunctionTok{fixef}\NormalTok{(fit2)[}\DecValTok{4}\NormalTok{] }\SpecialCharTok{\textless{}=}\NormalTok{ deltasim2}\SpecialCharTok{$}\NormalTok{upper) }\SpecialCharTok{|\textgreater{}}\NormalTok{ knitr}\SpecialCharTok{::}\FunctionTok{kable}\NormalTok{(}\AttributeTok{col.names =} \StringTok{" "}\NormalTok{)}
\end{Highlighting}
\end{Shaded}

\begin{longtable}[]{@{}r@{}}
\toprule\noalign{}
\endhead
\bottomrule\noalign{}
\endlastfoot
0.9465 \\
\end{longtable}

The estimates are still practically unbiased and the confidence
intervals achieve accurate coverage.

\subsubsection{Simulating from the exponential
distribution}\label{simulating-from-the-exponential-distribution}

The mean of an exponentially distributed random variable \(X\) with rate
equal to \(1\), is \(E(X) = 1\). And the variance of an is
\$\frac{1}{\lambda^2} = 1 \$. In order to achieve a mean of \(0\) and a
variance of \(\sigma^2\) we would therefore need to shift and scale
\(X \sim exp(1)\) with

\[\sigma^2= V(c \cdot X - k) = c^2  \Leftrightarrow c = \sigma\] and

\[0 = E(\sigma X - k) = \sigma - k \Leftrightarrow k = \sigma\]

We define the shift and scaling constants

\begin{Shaded}
\begin{Highlighting}[]
\NormalTok{c\_par }\OtherTok{\textless{}{-}}\NormalTok{ tauP}
\NormalTok{c\_class }\OtherTok{\textless{}{-}}\NormalTok{ tauC }
\NormalTok{c\_eps }\OtherTok{\textless{}{-}}\NormalTok{ sigma}
\end{Highlighting}
\end{Shaded}

\begin{Shaded}
\begin{Highlighting}[]
\NormalTok{deltasim3 }\OtherTok{\textless{}{-}} \FunctionTok{matrix}\NormalTok{(}\ConstantTok{NA}\NormalTok{,M,}\DecValTok{3}\NormalTok{)}

\ControlFlowTok{for}\NormalTok{ (i }\ControlFlowTok{in} \DecValTok{1}\SpecialCharTok{:}\NormalTok{M)\{}
\NormalTok{  B1 }\OtherTok{\textless{}{-}} \FunctionTok{rexp}\NormalTok{(}\AttributeTok{n =}\NormalTok{ n\_par, }\AttributeTok{rate =} \DecValTok{1}\NormalTok{) }\SpecialCharTok{*}\NormalTok{ c\_par }\SpecialCharTok{{-}}\NormalTok{ c\_par}
\NormalTok{  B2 }\OtherTok{\textless{}{-}} \FunctionTok{rexp}\NormalTok{(}\AttributeTok{n =}\NormalTok{ n\_class, }\AttributeTok{rate =} \DecValTok{1}\NormalTok{) }\SpecialCharTok{*}\NormalTok{ c\_class }\SpecialCharTok{{-}}\NormalTok{  c\_class}
\NormalTok{  eps }\OtherTok{\textless{}{-}} \FunctionTok{rexp}\NormalTok{(}\AttributeTok{n =}\NormalTok{ n\_eps, }\AttributeTok{rate =} \DecValTok{1}\NormalTok{) }\SpecialCharTok{*}\NormalTok{ c\_eps }\SpecialCharTok{{-}}\NormalTok{ c\_eps}
\NormalTok{  B }\OtherTok{\textless{}{-}} \FunctionTok{c}\NormalTok{(B1,B2)}
\NormalTok{  y }\OtherTok{\textless{}{-}}\NormalTok{ X }\SpecialCharTok{\%*\%}\NormalTok{ beta }\SpecialCharTok{+}\NormalTok{ Z }\SpecialCharTok{\%*\%}\NormalTok{ B }\SpecialCharTok{+}\NormalTok{ eps}
\NormalTok{  y }\OtherTok{\textless{}{-}}\NormalTok{ y }\SpecialCharTok{|\textgreater{}} \FunctionTok{as.numeric}\NormalTok{() }\CommentTok{\# NB. This seems to be necessary}
\NormalTok{  lmm2 }\OtherTok{\textless{}{-}} \FunctionTok{lmer}\NormalTok{(y }\SpecialCharTok{\textasciitilde{}}\NormalTok{ ProdVersion }\SpecialCharTok{+}\NormalTok{ ProdType }\SpecialCharTok{+}\NormalTok{ (}\DecValTok{1}\SpecialCharTok{|}\NormalTok{Participant) }\SpecialCharTok{+}\NormalTok{ (}\DecValTok{1}\SpecialCharTok{|}\NormalTok{Class), }\AttributeTok{data=}\NormalTok{likingdata)}
\NormalTok{  deltasim3[i,}\DecValTok{1}\NormalTok{] }\OtherTok{\textless{}{-}} \FunctionTok{fixef}\NormalTok{(lmm2)[}\DecValTok{4}\NormalTok{]}
\NormalTok{  deltasim3[i,}\DecValTok{2}\SpecialCharTok{:}\DecValTok{3}\NormalTok{] }\OtherTok{\textless{}{-}}\NormalTok{ (lmm2 }\SpecialCharTok{|\textgreater{}} \FunctionTok{confint}\NormalTok{(}\AttributeTok{method=}\StringTok{"Wald"}\NormalTok{))[}\DecValTok{7}\NormalTok{,]}
\NormalTok{\}}

\NormalTok{deltasim3 }\OtherTok{\textless{}{-}}\NormalTok{ deltasim3 }\SpecialCharTok{|\textgreater{}} \FunctionTok{data.frame}\NormalTok{()}
\FunctionTok{names}\NormalTok{(deltasim3) }\OtherTok{\textless{}{-}} \FunctionTok{c}\NormalTok{(}\StringTok{"est"}\NormalTok{,}\StringTok{"lower"}\NormalTok{,}\StringTok{"upper"}\NormalTok{)}
\end{Highlighting}
\end{Shaded}

We calculate the bias:

\begin{Shaded}
\begin{Highlighting}[]
\FunctionTok{mean}\NormalTok{(deltasim3}\SpecialCharTok{$}\NormalTok{est }\SpecialCharTok{{-}} \FunctionTok{fixef}\NormalTok{(fit2)[}\DecValTok{4}\NormalTok{]) }\SpecialCharTok{|\textgreater{}}\NormalTok{ knitr}\SpecialCharTok{::}\FunctionTok{kable}\NormalTok{(}\AttributeTok{col.names =} \StringTok{" "}\NormalTok{)}
\end{Highlighting}
\end{Shaded}

\begin{longtable}[]{@{}r@{}}
\toprule\noalign{}
\endhead
\bottomrule\noalign{}
\endlastfoot
-0.005931 \\
\end{longtable}

And the coverage:

\begin{Shaded}
\begin{Highlighting}[]
\FunctionTok{mean}\NormalTok{(deltasim3}\SpecialCharTok{$}\NormalTok{lower }\SpecialCharTok{\textless{}=} \FunctionTok{fixef}\NormalTok{(fit2)[}\DecValTok{4}\NormalTok{] }\SpecialCharTok{\&} \FunctionTok{fixef}\NormalTok{(fit2)[}\DecValTok{4}\NormalTok{] }\SpecialCharTok{\textless{}=}\NormalTok{ deltasim3}\SpecialCharTok{$}\NormalTok{upper) }\SpecialCharTok{|\textgreater{}}\NormalTok{ knitr}\SpecialCharTok{::}\FunctionTok{kable}\NormalTok{(}\AttributeTok{col.names =} \StringTok{" "}\NormalTok{)}
\end{Highlighting}
\end{Shaded}

\begin{longtable}[]{@{}r@{}}
\toprule\noalign{}
\endhead
\bottomrule\noalign{}
\endlastfoot
0.9505 \\
\end{longtable}

The estimates are still almost unbiased and the confidence achieve
accurate coverage. We can conclude that the model estimates and
confidence intervals are not too sensitive to the type of distribution
as long as the mean and variance is correctly specified.

\emph{plot evt. histogrammer}

\subsection{7.}\label{section-7}

We first examine the residuals. We do this by plotting the fitted values
at level one against the residuals.

\begin{Shaded}
\begin{Highlighting}[]
\FunctionTok{ggplot}\NormalTok{(}\AttributeTok{data =} \FunctionTok{data.frame}\NormalTok{(}\FunctionTok{fitted}\NormalTok{(fit2)), }\FunctionTok{aes}\NormalTok{(}\AttributeTok{x =} \FunctionTok{fitted}\NormalTok{(fit2), }\AttributeTok{y =} \FunctionTok{residuals}\NormalTok{(fit2)))}\SpecialCharTok{+}
  \FunctionTok{geom\_point}\NormalTok{()}\SpecialCharTok{+}
  \FunctionTok{geom\_smooth}\NormalTok{(}\AttributeTok{se =} \ConstantTok{FALSE}\NormalTok{)}
\end{Highlighting}
\end{Shaded}

\begin{verbatim}
## `geom_smooth()` using method = 'loess' and formula = 'y ~ x'
\end{verbatim}

\includegraphics{Aflevering_1_pdf_files/figure-latex/unnamed-chunk-21-1.pdf}
The lines are a result of the the response being discrete. The lines
make it seem as if there are trends in the residuals, but as the
geom\_smooth shows, the residuals have approximately mean \(0\).

QQ-plot:

\begin{Shaded}
\begin{Highlighting}[]
\NormalTok{fit2 }\SpecialCharTok{|\textgreater{}} \FunctionTok{resid}\NormalTok{() }\SpecialCharTok{|\textgreater{}} \FunctionTok{qqnorm}\NormalTok{(}\AttributeTok{main=}\StringTok{""}\NormalTok{)}
\end{Highlighting}
\end{Shaded}

\includegraphics{Aflevering_1_pdf_files/figure-latex/unnamed-chunk-22-1.pdf}
Looks good. We now turn to inspect the predicted random effects. They
look OK.

\begin{Shaded}
\begin{Highlighting}[]
\FunctionTok{library}\NormalTok{(lattice)}
\NormalTok{fit2 }\SpecialCharTok{|\textgreater{}} \FunctionTok{ranef}\NormalTok{() }\SpecialCharTok{|\textgreater{}} \FunctionTok{dotplot}\NormalTok{()}
\end{Highlighting}
\end{Shaded}

\begin{verbatim}
## $Participant
\end{verbatim}

\includegraphics{Aflevering_1_pdf_files/figure-latex/unnamed-chunk-23-1.pdf}

\begin{verbatim}
## 
## $Class
\end{verbatim}

\includegraphics{Aflevering_1_pdf_files/figure-latex/unnamed-chunk-23-2.pdf}

\begin{Shaded}
\begin{Highlighting}[]
\FunctionTok{qqmath}\NormalTok{(}\FunctionTok{ranef}\NormalTok{(fit2)}\SpecialCharTok{$}\NormalTok{Participant[,}\DecValTok{1}\NormalTok{], }\AttributeTok{main=}\StringTok{"Participant"}\NormalTok{)}
\end{Highlighting}
\end{Shaded}

\includegraphics{Aflevering_1_pdf_files/figure-latex/unnamed-chunk-23-3.pdf}

\begin{Shaded}
\begin{Highlighting}[]
\FunctionTok{qqmath}\NormalTok{(}\FunctionTok{ranef}\NormalTok{(fit2)}\SpecialCharTok{$}\NormalTok{Class[,}\DecValTok{1}\NormalTok{], }\AttributeTok{main=}\StringTok{"Class"}\NormalTok{)}
\end{Highlighting}
\end{Shaded}

\includegraphics{Aflevering_1_pdf_files/figure-latex/unnamed-chunk-23-4.pdf}

Maybe some simulation to conclude whether the plots looks like they
should.

\subsection{8.}\label{section-8}

In order to examine how discretization of the response affects the
LMM-based estimator for \(\delta\), we simulate from the fit2 model,
discretize the simulations by using the \texttt{round} function, fit a
model on the simulated discretized data and extract the estimates and
the CI boundaries.

\begin{Shaded}
\begin{Highlighting}[]
\NormalTok{deltasim4 }\OtherTok{\textless{}{-}} \FunctionTok{matrix}\NormalTok{(}\ConstantTok{NA}\NormalTok{,M,}\DecValTok{3}\NormalTok{)}

\ControlFlowTok{for}\NormalTok{ (i }\ControlFlowTok{in} \DecValTok{1}\SpecialCharTok{:}\NormalTok{M)\{}
\NormalTok{  y }\OtherTok{\textless{}{-}} \FunctionTok{simulate}\NormalTok{(fit2)}\SpecialCharTok{$}\NormalTok{sim\_1 }\SpecialCharTok{|\textgreater{}} \FunctionTok{round}\NormalTok{() }\CommentTok{\# evt. put alle forudsigelser større end 7 til 7, er det derfor vi har problemer i mange simulationer?????}
\NormalTok{  lmm2 }\OtherTok{\textless{}{-}} \FunctionTok{lmer}\NormalTok{(y }\SpecialCharTok{\textasciitilde{}}\NormalTok{ ProdVersion }\SpecialCharTok{+}\NormalTok{ ProdType }\SpecialCharTok{+}\NormalTok{ (}\DecValTok{1}\SpecialCharTok{|}\NormalTok{Participant) }\SpecialCharTok{+}\NormalTok{ (}\DecValTok{1}\SpecialCharTok{|}\NormalTok{Class), }\AttributeTok{data=}\NormalTok{likingdata)}
\NormalTok{  deltasim4[i,}\DecValTok{1}\NormalTok{] }\OtherTok{\textless{}{-}} \FunctionTok{fixef}\NormalTok{(lmm2)[}\DecValTok{4}\NormalTok{]}
\NormalTok{  deltasim4[i,}\DecValTok{2}\SpecialCharTok{:}\DecValTok{3}\NormalTok{] }\OtherTok{\textless{}{-}}\NormalTok{ (lmm2 }\SpecialCharTok{|\textgreater{}} \FunctionTok{confint}\NormalTok{(}\AttributeTok{method=}\StringTok{"Wald"}\NormalTok{))[}\DecValTok{7}\NormalTok{,]}
\NormalTok{\}}

\NormalTok{deltasim4 }\OtherTok{\textless{}{-}}\NormalTok{ deltasim4 }\SpecialCharTok{|\textgreater{}} \FunctionTok{data.frame}\NormalTok{()}
\FunctionTok{names}\NormalTok{(deltasim4) }\OtherTok{\textless{}{-}} \FunctionTok{c}\NormalTok{(}\StringTok{"est"}\NormalTok{,}\StringTok{"lower"}\NormalTok{,}\StringTok{"upper"}\NormalTok{)}
\end{Highlighting}
\end{Shaded}

We calculate the bias:

\begin{Shaded}
\begin{Highlighting}[]
\FunctionTok{mean}\NormalTok{(deltasim4}\SpecialCharTok{$}\NormalTok{est }\SpecialCharTok{{-}} \FunctionTok{fixef}\NormalTok{(fit2)[}\DecValTok{4}\NormalTok{]) }\SpecialCharTok{|\textgreater{}}\NormalTok{ knitr}\SpecialCharTok{::}\FunctionTok{kable}\NormalTok{(}\AttributeTok{col.names =} \StringTok{" "}\NormalTok{)}
\end{Highlighting}
\end{Shaded}

\begin{longtable}[]{@{}r@{}}
\toprule\noalign{}
\endhead
\bottomrule\noalign{}
\endlastfoot
-0.0004756 \\
\end{longtable}

And coverage:

\begin{Shaded}
\begin{Highlighting}[]
\FunctionTok{mean}\NormalTok{(deltasim4}\SpecialCharTok{$}\NormalTok{lower }\SpecialCharTok{\textless{}=} \FunctionTok{fixef}\NormalTok{(fit2)[}\DecValTok{4}\NormalTok{] }\SpecialCharTok{\&} \FunctionTok{fixef}\NormalTok{(fit2)[}\DecValTok{4}\NormalTok{] }\SpecialCharTok{\textless{}=}\NormalTok{ deltasim4}\SpecialCharTok{$}\NormalTok{upper) }\SpecialCharTok{|\textgreater{}}\NormalTok{ knitr}\SpecialCharTok{::}\FunctionTok{kable}\NormalTok{(}\AttributeTok{col.names =} \StringTok{" "}\NormalTok{)}
\end{Highlighting}
\end{Shaded}

\begin{longtable}[]{@{}r@{}}
\toprule\noalign{}
\endhead
\bottomrule\noalign{}
\endlastfoot
0.949 \\
\end{longtable}

We furthermore make a histogram over the estimated values

\begin{center}\includegraphics{Aflevering_1_pdf_files/figure-latex/unnamed-chunk-27-1} \end{center}

We still obtain unbiased estimates and accurate coverage. The
distribution looks approximately normal as well\ldots{}

\end{document}
